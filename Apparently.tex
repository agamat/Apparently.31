																												% Thesis Preamable
%%%%%%%%%%%%%%%%%%%%%%%%%%%%%%%%%%%%%%%%%%%%%%%%%%%%%%%%%%%%%%%%%%%%
                                          																			 % Packages																									%	
\documentclass[12pt]{article}																																															%
\usepackage{graphicx} 						  	 % Add pictures to your document																												%					
\usepackage{lscape}   								 % Use landscape in one page																														%
\usepackage{mfirstuc} 							   % Conatins Capitalize each word																												%
\usepackage{amsmath}                             % This is for math writing																															%
\usepackage[strings]{underscore}  			 % This is to recoginse the underscore																										%
\usepackage{verbatim}
\usepackage{apacite}                          		 % The citation package.																																%	
\usepackage[english]{babel} 					 % Use all English Language																														%
\setlength{\parindent}{2em}					     % Paragraph indent pakacge																														%
\usepackage{indentfirst}					    	 % To indent first paragraph of each section     																							%
\usepackage{tikz}
\usepackage{cite}
\usepackage{natbib}								 	 % Gives different bib styles																														%																																												\usepackage{tikz}																													%																																																																									%																																																																									%																																																																									%																																																																									%																																																																									%
                                          																				 % Cover Page																							%
%\author{Ahmed F. Salhin}																																																	%
\title{Investor Sentiment and Stock Prices: Further Evidence from the UK}																							%
\date{\today{}}																																																					%
%%%%%%%%%%%%%%%%%%%%%%%%%%%%%%%%%%%%%%%%%%%%%%%%%%%%%%%%%%%%%%%%%%%%


\begin{document}
	\maketitle{}
	\pagenumbering{gobble}
	\pagenumbering{arabic}


\section{Introduction}

\cite{Baker2007a} defined investor sentiment as “a belief about future cash flows and investment risks that is not justified by the facts at hand”. If investors base their investment decisions on this belief, asset prices will be driven away from their fundamental values. Prices then might be consistently over or underpriced as a result of the limited and costly arbitrage that would bring prices back to equilibrium. Investors sentiment can cost the market by increasing probability of bubbles occurrence. For example, Alan Greenspan, former chairman of the US Federal Reserve uses the term “Irrational Exuberance” to express his warning of the internet bubble. This link between investor irrtionality and financial crises is supported by \cite{Zouaoui2011} who show that the existence of investor sentiment increases the likelihood of stock market crises occurrence. 

\par Studies on investors irrationallity aim at invistigating the relationship between stock market returns and changes in invetor sentiment. While invistigating this relationship, these studies provides twofold benifits. The first is either the support or the creation of a measure of sentiment since investor sentiment is unobservabel. The second is the identfication of shape of the relationship between investor sentiment and stock returns. Examples of these studies are \cite{Solt1988}, \cite{Lee1991,Abraham1993,Neal1998,Eichengreen1998,Fisher2000,Wang2003}; \cite{Baker2006, Chang2007,Beaumont2008}.

\par Some measures of investor sentiment have been used in exploring its impact on the financial market such as closed-end fund discount \cite{Lee1991}, investor and consumer surveys \cite{Otoo1999, Jansen2003, Schmeling2009}, mutual fund flows \cite{Brown2003}, and a composite sentiment indexes that combine some of those has been mentioned and others \cite{Brown2004, Baker2006}. Using these measure, studies found that stock market returns are affected by the aggregate investor sentiment in the market. However, this approach assumes that investors hold the same sentiment towards different asset classes, stocks from different sectors  and stocks with different characteristics which might, if invistigated, result in a distingusied impact of investor sentiment on stock returns. In order to invistigate any of these relationships, data on lower level than the aggregate market's sentiment is required.

\par Using monthly EU Commision Business Climate Indicators that covers the period from January 1985 to October 2014, this study attempts to answer the follwoing two questions; 1-Is the relationship between investor sentiment and asset prices holds for all stock market sectors? 2- Which sector contributes the most towards the aggregate relationship between investor sentiment and stock prices? The following section reviews the literature on the role investor sentiment in the financial market. Section three discusses the data and the methdoogy used in the study. Empirical results are discussed in details in section four and section five concludes.

\section{Investor Sentiment and Stock Returns}
The impact of investeor sentiment on stock market return was a subject of a debate for a long time. That debate has three main parts; one is about whether the relationship is exist or not, how to measure investor sentiment and if the relationship is exist and senitment is captured by some indices, how useful these indices for invistors. Although the recent researches on the field are invistigateing much more beyond that debte, it worths breifly refering to these parts of the debate.
\par Economics literatures discussed the individual choices and how they are made in a rational context. For many theories in the field and specially in financial economics, this individual rationality when making choices is an important assumption to understand the behaviour of both agents and prices in the market. This assumption is also hold even for studies that developed toughts of individuals' normal way of making and taking decisions such as \cite{Thaler1981}. In there theroy of self control, they viewed individual to perform two conflicting roles at the same point of time; one is planner and the other is selfish doer, however, both are a result of a rational choice to play both roles. 

\par Efficient Market Hypothesis (EMH) by Fama (1970) is one of the theroies that assumes investor rationality. It claims the prices in the market are right and reflects every available information about the asset. If, for any reason, prices are driven away from their fundamental values, arbitragures are wise enough to exploit the difference between current and fundamental prices and set the price back to its intrnsic value. On the other hand, early studies of investor sentiment such as \cite{Lee1991, Sanders1997} and \cite{DeBondt1998} consider arbitrage to be costly and not profitable, therefore, assets will be consistently mispriced. Combining both arguments (i.e. the first one that arbitrage is not needed because of the investor is rational and prices are right and arbitrage is not profitable and therefore not excuted accoding to the second one ) lefts no confirmation whether investor is rational or not but it concentrate mainly on the arbitragures ability to reset prices to equilibrium. 
\par Nevertheless, some anomalies in the financial market can not be understood under the traditional Efficient Market Hypothesis. One of these anomalies is the closed-end funds discount puzzle. Closed-end funds are a traded class of assets that their shares' price is determend by the forces of demand and supply. These funds invest in unique/different classed of assets (which themselves have a prices in the market) along with bearing some liablities which results in a net value called "net asset value" or NAV. According to the EMH, closed-end funds share price should almost equal its NAV per share to reflect all availabe information about the fund, however, their is always a difference called discount.
\par \cite{Lee1991} argue that closed-end funds discount is a reflection to investors sentiment in the market. They found that movement of closed-end funds discount is correlated with small companies' stockes held by indvidual investors whose decisions are more likely to be affected by sentiment. This is supported also by the work of \cite{GregoryM.Noronha1995, Chen2003, Chen1993, JamesN.BodurthaDong-SoonKim1995}.

(to be completed)


\section{Data}
This study used sentiment indicators published by the European Commission as a proxy for investor sentiment.  The indicators are calculated from business and consumer surveys that conducted on a monthly bases by national institutions (such as ministries, statistical offices, central banks, research institutes, business associations or private companies) in 27 European countries. In every country, businesses and consumers are surveyed for their opinions regarding the economic conditions and short term forcasting. The surveys are then harmonized for to generate comparable data for the countries have been surveyed. 
\par For business indicators, five surveys are conducted on a monthly basis with more questions added on a quarterly bases. The surveys cover manufacturing industry, construction, retail trade, services, and financial services. A biannually investment survey of the manufacturing sector is conducted to gathers information on companies’ investment plans. Classification of business surveys into sectors follows the Classification of economic activities in the European Community (NACE). 
\par All surveys used in this study cover the period from January 1985 to October 2014 except surveys on services and financial services sectors. Surveys on services sector starts from January 1997 with the inclusion of financial services companies until it covered by a separte survey starting from April 2006. For the purpose of this study, general services sector data has been used because of the lack of avalablility of financial services sector data. Sample size of each sector is in Table[1].
\par \vspace{0.1 cm}
\begin{figure}[!htb]
	{\centering
		\tablename{{[1]: Businesses Surveys Sample Size}}
	\begin{tabular}{|c|c|c|c|c|}
			\hline  Sectors & Manufacturing Industry & Services & Retail trade  & Construction\\ 
			\hline UK & 1500 & 1000 & 500  & 750 \\ 
			\hline EU & 38,270 & 43,720 & 30,730 & 22,140  \\ 
			\hline NI & CBI & CBI & CBI & EXP.  \\
			\hline
		\end{tabular} 
		\par \noindent \footnotesize  NI (National Institution), CBI (The Confideration of British Industry), EXP (Experian)} \textit {Source: European Commission services}
\end{figure}

\par  Monthly surveys are performed in the first two to three weeks of each month for all business and consumer indicators. This has been matched by the data on stock prices which will be discussed later in this section. Questions of surveys used in the study has mainly qualitative replies with a three, five or six ordinal scale. Example of replies are “increase”, “remain unchanged”, “decrease”; or “more than sufficient”, “sufficient”, “not sufficient”; or “too large”, “adequate”, “too small”. Sample of questions is in the appendix.
\par Confidence Indicators are calculated using balances that summarize replies to surveys questions. Percentage of responds to any single question should follow
{\par \centering $PP+P+E+N+NN+M=100$\\}
\vspace{0.2 cm}
\par while $PP$ is very positive, $P$ is positive, $E$ is neutral, $N$ is negative, $NN$ is very negative and $M$ is without any opinion.

\par Balances then are calculated as:
\vspace{0.5 cm}
{\par \centering $B = (PP + \frac{1}{2}P) - (\frac{1}{2}N + NN)$\\}
\vspace{0.5 cm}
the balance of a question is ranged from -100 all respondents choose the negative option to +100 all respondents choose the positive option. Also they are seasonally adjusted using "Dainties" as the seasonal-adjustment algorithm. For each sector, Confidence Indicator is the simple arithmatic average of all seasonally adjusted balances of questions.

\par The aggregate sentiment indicator of the market named as Economic Sentiment Indicator (ESI) is the weighted average of all Confidience Indicators with 40 \% to manufacturing, 30\% to services, 20 \% consumers, 5\% for each of construction and retail trade sectors. Values of Confidence Indicator along with Economic Sentiment Indicator are plotted if Figure[1].

%\begin{comment}
\begin{figure}[!htb]
% The output from tikz()
{\centering
% is imported here.
\caption{UK Confidence and Economic Sentiment Indicators}
\input{/Users/Ahmed/Documents/Apps/mmukcc.tex}}
\end{figure}
%\end{comment}


Our analysis covers both the relationship between the aggregate market sentiment and stock prices and also on a sector level. FTSE100 Index monthly prices have been used for the aggregate level and calculated as the average of the first 21 days of the month to match the period of conducting confidence indicator surveys (also average of 30/31 days has been used, however, there is not difference in the final results). 

\par Sector prices has been obtained by classifying FTSE All-Shares constituents into sectors according to the Industry Classification Benchmark (ICB) to match the Classification of economic activities in the European Community (NACE) mentioned above. Sector monthly prices is calculated in the same way as in FTSE100 (i.e. for 21 and 30/31 days), however, the value of daily prices of the sector represents prices of sector companies weighted by the market value of each company. Both FTSE100 and FTSE All-Shares data are obtained from Datastream for the period from January 1985 to October 2014.  For the service sector, prices for the period from January 1997 to April 2006 covers both the general services and financial services and the later has been removed from the sample afterward to match services confidence indicator data. Number of companies in every sectors is 212 in manufacturing, 13 in construction, 34 in retail trade, 87 in services and 278 in financial services. 

\par For the analysis purpose, Augmented Dickey-Fuller test has been used to test for stationarity. Test confirmed that both the log of the stock prices and the confidence indicators are all I(1) variables for the aggregate market and all sectors. Results of the test is available in the Appendix.

\section{Methodology and Findings}

Our findings is based on the results of Granger-Causality test to capture whether a time series (confidence indicator) can forecast another time series which in our case stock prices. The model we employed has the form:

 \begin{equation*} \label{eq1}
\begin{split}
\Delta P_t= \alpha_{p} + \sum_{i=1}^{k} \beta_{pi} \Delta P_{t-1}+ \sum_{i=1}^{k} \gamma_{pi} \Delta S_{t-1}+ \upsilon_{pt} 
\end{split}
\end{equation*}

\par while $P_t$ is the log of price index of sectors and market at time $t$, $S_t$ denotes sector confidence indicator/economic sentiment indicator at time $t$,  $\Delta$  denotes the first difference and $\upsilon$ is a disturbance, and k is the maximum lag.
\par This model means that sentiment is believed to Granger-cause stock prices if lagged sentiment contain information that is not already contained in past values of stock returns. The model is applied to test the following null hypothesis for the aggregate market and for each of the sectors:
\vspace{0.3 cm}
\\ $H_{0}$: Sentiment does not Granger-cause prices 
\vspace{0.3 cm}
\par Table[3] summarize both Granger Causality test and Wald-type test used to test for nonzero correlation between the error processes of the sentiment and prices.

\begin{figure}[!htb]
	{\centering
		\begin{flushleft}
		\tablename{[3]: Granger Causality and Wald-type tests}
		\end{flushleft}}
	\begin{tabular}{|c|c|c|c|c|}
		\hline & Granger Causality & Contemporaneous & Wald-type test\\ 
		& \textit p-value & Correlation  & \textit p-value\\ 
		\hline Market &  0.5612 & $0.20^{*}$ & 0.000  \\ 
		Manufacturing & 0.8929 & 0.06 & 0.321  \\ 
		Construction & 0.7912 & 0.08 & 0.089   \\
		Services & 0.2377 & -0.02 & 0.984  \\
		Retail Trade & 0.7909 & $0.15^{*}$ & 0.012   \\
		\hline 
	\end{tabular} 
	\begin{flushleft}
		\footnotesize  * is significant at 10 \% level
	\end{flushleft}
\end{figure}

\par According to the Granger Causality results as shown in Table[3], we failed to reject the null hypothesis that sentiment does not Granger-cause stock prices for the aggregate market and for every sector. 
(to be completed)


\newpage

\bibliography{/Users/Ahmed/Google Drive/Apps/Tex/library.bib}
\bibliographystyle{apalike}

\verbatiminput{adfgc.txt}


\end{document}